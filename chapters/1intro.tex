The evolution of perception systems across species, from insects to apex predators, underscores nature's prowess. Human technological innovation has given rise to mechanical eyes that not only emulate but sometimes surpass their biological counterparts. This union of natural instincts and modern camera technology unleashes a vast realm of perception. Central to computer vision is the light spectrum, extending from visible colors (around 400 to 700 nanometers) to the invisible realms of infrared and ultraviolet waves. While humans are confined to a segment of this spectrum, technology unlocks a wider expanse, particularly through visible and infrared imagery, ushering in a revolutionary phase in computer vision. Image fusion, a pivotal aspect of computer vision, combines multiple image sources to enhance clarity and understanding, with Visual-Infrared Image Fusion (VIF) proving transformative for applications such as night vision and thermal imaging. Various techniques, from traditional pixel and feature-level fusion to advanced methods like CNN-based and transformer-based fusion, are employed to merge visible and infrared images effectively, amplifying situational comprehension across applications.

The advancement of deep learning methods in extracting pertinent features for image fusion necessitates addressing their primary focus on local features, often neglecting the overarching context within the image, as highlighted by Li et al. (2021) \cite{li2021rfn}. Transformer-based models, leveraging attention mechanisms, offer a solution by capturing wider context dependencies to refine fusion outcomes.

A significant challenge in fusion techniques lies in crafting an appropriate loss function. Conventional methods heavily rely on the Structural Similarity Metric (SSIM) to ascertain similarity between the input visible band image and the fused output. This SSIM-centric approach has limitations, given its maximum value is achieved only when both images are identical. To address this, we propose a novel loss function that equally emphasizes congruence with both the visible band and infrared inputs, ensuring optimal retention of essential information and bolstering fusion quality.

In response to the need for enhanced image fusion techniques, our research focuses on several key questions. Firstly, we explore the hypothesis that the new loss function, emphasizing the similarity between both the visible band input and the infrared image input, will result in more informative and meaningful fused images compared to using Structural Similarity Metric (SSIM) as the guiding criterion \cite{li2021rfn}. Secondly, we aim to determine if the proposed approach will surpass existing state-of-the-art image fusion methods in terms of visual quality, providing more detailed, sharper, and visually appealing fused images \cite{li2021rfn}. Additionally, we investigate whether the combination of Transformer-based models and the new loss function will significantly improve night vision enhancement, medical imaging, and surveillance tasks, allowing for better object detection, classification, and tracking in challenging lighting conditions \cite{dosovitskiy2020image, liu2021swin}. Furthermore, we explore whether the proposed transformer-based approach will achieve a better balance between quantitative and qualitative performance in image fusion, overcoming the compromise between the two often observed in traditional deep learning methods \cite{dosovitskiy2020image, liu2021swin}. Lastly, we assess whether the proposed approach will demonstrate computational efficiency, making it suitable for real-time applications, such as video surveillance and live medical imaging, without sacrificing the quality of the fused images \cite{dosovitskiy2020image, liu2021swin}. Through these research questions, we aim to gain a comprehensive understanding of the complexities surrounding image fusion and contribute to advancements in the field.